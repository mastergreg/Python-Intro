\documentclass{beamer}
\usepackage{minted}
\usetheme{Dresden}
\author[nemo/mastergreg]{Greg (mastergreg) Liras, John (nemo) Giannelos}
\institute{FOSS NTUA}
\title{Python Tutorial}
\subtitle{Part I}

\date{\today}


\begin{document}
\AtBeginSection[]
{
\begin{frame}
	\frametitle{Outline}
	\tableofcontents[currentsection]
\end{frame}
}
%--- the titlepage frame -------------------------%
\begin{frame}
	\titlepage
\end{frame}


\section{Introduction to Python}

\subsection{What is Python?}
\begin{frame}
	\frametitle{What is Python?}
\begin{quote}
Python is an easy to learn, powerful programming language. It has efficient
high-level data structures and a simple but effective approach to
object-oriented programming. Python’s elegant syntax and dynamic typing,
together with its interpreted nature, make it an ideal language for scripting
and rapid application development in many areas on most platforms.
\end{quote}
\end{frame}

\subsection{Freatures}
\begin{frame}
	\frametitle{Python Tutorial}

In a few words, Python,
\begin{itemize}
\item<1-> is \emph{Scripting Language}
\item<2-> is \emph{Strongly Typed}
\item<3-> is \emph{Dynamic}
\item<4-> is \emph{Portable}
\item<5-> is \emph{Object Oriented}
\item<6-> has \emph{Vast Libraries}
\item<7-> is \emph{Simple and non-obtrucive}
\end{itemize}

\end{frame}

\subsection{Why Python?}
\begin{frame}
	\frametitle{Why?}
	\begin{itemize}
	\item<1-> It is easy to remember
	\item<2-> You can develop rapidly
	\item<3-> Interface with C libraries
	\end{itemize}
\end{frame}
\subsection{Dos and Don'ts}
\begin{frame}
	\frametitle{Must and Must Not}

\begin{itemize}
\item<1-> Search first code less
\item<2-> Import only what you need
\item<3-> Run pychecker on your code
\end{itemize}

\end{frame}

\section{Python Standard Types}
\subsection{Arithmetic}

\begin{frame}
\frametitle{Numeric types}
\begin{itemize}
\item<1-> int (up to $10^{308}$!!!!) 
\item<2-> float (53 bits precision)
\item<3-> complex ($1+2j$)
\end{itemize}
\end{frame}

\begin{frame}
\frametitle{Operators}
\begin{itemize}
\item<1-> + (add)
\item<2-> - (subtract)
\item<3-> * (multiply)
\item<4-> / (divide)
\item<5-> \% (modulo)
\item<6-> =	(assign)
\end{itemize}
\end{frame}

\subsection{Strings}
\begin{frame}[fragile]
\frametitle{Strings}
\begin{itemize}
\item<1-> Strings are not lists! Strings are immutable!
\item<2-> Simple concatenation: 
\begin{minted}[fontsize=\scriptsize]{python}
>>> 'Hello' + 'World'
'HelloWorld'
\end{minted}
\item<3-> Slicing:
\begin{itemize}
\item<4->\begin{minted}[fontsize=\scriptsize]{python}
>>> 'HelloWorld'[0]
'H'
\end{minted}
\item<5->\begin{minted}[fontsize=\scriptsize]{python}
>>> 'HelloWorld'[6:]
'orld'
\end{minted}
\end{itemize}
\item<6-> Unicode Strings:
\begin{minted}[fontsize=\scriptsize]{python}
>>> ur'Hello\u0020World !'
u'Hello World !'
\end{minted}
\end{itemize}
\end{frame}

\subsection{Data Structures}

\begin{frame}[fragile]
\frametitle{Lists}
\begin{itemize}

\item<1-> \begin{minted}[fontsize=\scriptsize]{python}
>>> a = ['spam', 'eggs', 100, 1234]
>>> a
['spam', 'eggs', 100, 1234]
\end{minted}

\item<2-> Negative indices:
\begin{minted}[fontsize=\scriptsize]{python}
>>> a[-2]
100
\end{minted}

\item<3-> Concatenation:
\begin{minted}[fontsize=\scriptsize]{python}
>>> a[:2] + ['bacon', 2*2]
['spam', 'eggs', 'bacon', 4]
\end{minted}

\item<4-> Comprehension:
\begin{minted}[fontsize=\scriptsize]{python}
for i in a:
    print i
\end{minted}

\end{itemize}
\end{frame}
\begin{frame}
\frametitle{Tuples}
\begin{itemize}
\item<1-> Immutable (just as strings)
\item<2-> Indexed
\item<3-> Nested
\end{itemize}
\end{frame}

\begin{frame}[fragile]
\frametitle{Sets}
\begin{quote}
A set is an unordered collection with no duplicate elements. 
\end{quote}
\begin{itemize}
\item<2-> \begin{minted}[fontsize=\scriptsize]{python}
>>> basket = ['apple', 'orange', 'apple', 'pear', 'orange', 'banana']
>>> set(basket)
set(['orange', 'pear', 'apple', 'banana'])
\end{minted}
\item<3-> Operators: 
\begin{itemize}
\item<4-> a - b (in a but not in b)
\item<5-> a $\vert$ b (in a or in b)
\item<6-> a \& b (in a and in b)
\item<7-> a \textasciicircum  b (in a or b but not in both)
\end{itemize}
\end{itemize}
\end{frame}

\begin{frame}[fragile]
\frametitle{Dictionaries}
Maps of objects
\begin{itemize}
\item<2-> Easy to create \begin{minted}[fontsize=\scriptsize]{python}
>>> dict([('sape', 4139), ('guido', 4127), ('jack', 4098)])
{'sape': 4139, 'jack': 4098, 'guido': 4127}
\end{minted}
\item<3-> Simple to use \begin{minted}[fontsize=\scriptsize]{python}
>>> tel = dict([('sape', 4139), ('guido', 4127), ('jack', 4098)])
>>> tel['jack']
4098
\end{minted}
\end{itemize}
\end{frame}


\end{document}
